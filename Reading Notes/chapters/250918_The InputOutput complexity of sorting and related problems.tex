\section{The Input/Output complexity of sorting and related problems}

\textbf{Author(s):} Alok Aggarwal and Jeffrey Scott Vitter

\subsection{Paper notes}

The motivation for this paper is drawn from the real-world need for extremely fast sorting algorithms over extremely large datasets (on the order of tens of millions of records). 1/4 of all computer cycles are taken up by sorting, and most of those cycles are taken up by I/O between internal memory and secondary storage. 

The problem can be solved by relaxng the program requirements and introducing parallel/distributed processing, or to improve our understanding of the actual, theoretical limits underpinning sorting. The authors take the latter approach and set up the problem with the following parameters: $N$ records $(R_1, R_2, ..., R_N)$ to sort, $M$ records that can fit into internal memory, $B$ records per block, and $P$ blocs that can be transferred simultaneously. With this setup, records can be transferred in parallel within each block, and blocks can also be transferred in parallel. 

This setup is used in order to find theoretically optimal bounds for five problems: sorting, FFT, permutation networks, permuting, matrix transposition. 

\textbf{Sorting: } internal memory is empty