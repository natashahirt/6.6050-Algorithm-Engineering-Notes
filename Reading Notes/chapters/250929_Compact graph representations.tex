\section{An Experimental Analysis of a Compact Graph Representation}

\textbf{Author(s):} Daniel K. Blandford, Guy E. Blelloch, Ian A. Kash

\subsection{General Notes}
\begin{itemize}
    \item Storing large graphs for modern applications (social networks etc.)
    \item Unlabeled graphs (ordering on vertices can be picked by representation; need to store vertex labels separately for labeled graphs)
    \item Representation based on graph separators in previous work (most real world graphs have small separators)
    \item This paper: how does it stack up in practice? + how to handle dynamic graphs? (edge insertions, deletions), complete set of experiments
\end{itemize}

Comparison for static graphs: adjacency array
\begin{itemize}
    \item Adjacency array stores array of pointers to neighbors
    \item Concatenated into large array with each vertex pointing to beginning of own block
    \item Factor of 2 less space than adjacency list
    \item Use four codes for encoding differences: gamma codes, snip codes, nibble codes, byte codes
\end{itemize}

Comparison for dynamic representation: optimized implementation of adjacency lists
\begin{itemize}
    \item Optimized for space
    \item Optimized for time
    \item Performance is based on size of blocks used for storing the data and (for adjacency lists) ordering of vertices v.s. separator based which does not depend on insertion order
\end{itemize}

% Add your general notes here
