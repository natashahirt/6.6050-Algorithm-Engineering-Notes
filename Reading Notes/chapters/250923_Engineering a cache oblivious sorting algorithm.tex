\section{Engineering a cache-oblivious sorting algorithm}

\textbf{Author(s):} Gerth Stolting Brodal, Rolf Fagerberg, Kristoffer Vinther

\subsection{General Notes}

Many results have built on Friggo et al.
\begin{itemize}
    \item Often proven under the tall-cache assumption $M \geq B^2$
    \item Many theoretical results but few empirical (those that do show that cache-oblivious algorithms > classic RAM and are competitive with cache-aware algorithms)
    \item k-merger does not need a specific memory layout. Sizes of buffers not layout in memory are critical feature
\end{itemize}

This paper
\begin{itemize}
    \item Cache-oblivious sorting algorithm (faster than Quicksort for input sizes that fit into memory)
\end{itemize}

\subsection{Paper Review}

This paper presents an implemented version of Frigo et al.'s cache-oblivious search. One of the first points that they address is the theoretical optimality of using a two-level I/O model for multilevel memory hierarchies. They argue that if I/O operations are made by optimal cache replacement strategies, then the analysis holds for all levels of multilevel memory, and the algorithm will be optimized to all levels of cache.